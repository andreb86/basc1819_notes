%-------------------------------------------------------------------------------
%	PACKAGES AND OTHER DOCUMENT CONFIGURATIONS
%-------------------------------------------------------------------------------

\documentclass[paper=a4, fontsize=11pt]{report} % A4 paper and 11pt font size

\usepackage[T1]{fontenc} % Use 8-bit encoding that has 256 glyphs
\usepackage{fourier} % Use the Adobe Utopia font for the document - comment 
%this line to return to the LaTeX default
\usepackage[english]{babel} % English language/hyphenation
\usepackage{amsmath,amsfonts,amsthm} % Math packages
\usepackage{graphicx}
\usepackage{glossaries}
%\usepackage{floats}
%\graphicspath{{./pics/}}
\usepackage{sectsty} % Allows customizing section commands

\usepackage{fancyhdr} % Custom headers and footers
\usepackage{marvosym}
%-------------------------------------------------------------------------------
% Setup the listings package for the code excerpts
\usepackage{listings}
\usepackage{color}

\definecolor{mygreen}{rgb}{0,0.6,0}
\definecolor{mygray}{rgb}{0.5,0.5,0.5}
\definecolor{mymauve}{rgb}{0.58,0,0.82}

\lstset{ 
	backgroundcolor=\color{white},   % choose the background color; you must 
									 %add \usepackage{color} or 
									 %\usepackage{xcolor}; should come as last 
									 %argument
	basicstyle=\footnotesize,        % the size of the fonts that are used for 
									 %the code
	breakatwhitespace=false,         % sets if automatic breaks should only 
									 %happen at whitespace
	breaklines=true,                 % sets automatic line breaking
	captionpos=b,                    % sets the caption-position to bottom
	commentstyle=\color{mygreen},    % comment style
	deletekeywords={...},            % if you want to delete keywords from the 
									 %given language
	escapeinside={\%*}{*)},          % if you want to add LaTeX within your code
	extendedchars=true,              % lets you use non-ASCII characters; for 
									 %8-bits encodings only, does not work with 
									 %UTF-8
	frame=single,	                 % adds a frame around the code
	keepspaces=true,                 % keeps spaces in text, useful for keeping 
									 %indentation of code (possibly needs 
									 %columns=flexible)
	keywordstyle=\color{blue},       % keyword style
	language=[x86masm]Assembler,     % the language of the code
	morekeywords={*,...},            % if you want to add more keywords to the 
									 %set
	numbers=left,                    % where to put the line-numbers; possible 
									 %values are (none, left, right)
	numbersep=5pt,                   % how far the line-numbers are from the 
									 %code
	numberstyle=\tiny\color{mygray}, % the style that is used for the 
									 %line-numbers
	rulecolor=\color{black},         % if not set, the frame-color may be 
									 %changed on line-breaks within not-black 
									 %text (e.g. comments (green here))
	showspaces=false,                % show spaces everywhere adding particular 
									 %underscores; it overrides 
									 %'showstringspaces'
	showstringspaces=false,          % underline spaces within strings only
	showtabs=false,                  % show tabs within strings adding 
									 %particular underscores
	stepnumber=2,                    % the step between two line-numbers. If 
									 %it's 1, each line will be numbered
	stringstyle=\color{mymauve},     % string literal style
	tabsize=2,	                     % sets default tabsize to 2 spaces
	title=\lstname                   % show the filename of files included with 
									 %\lstinputlisting; also try caption 
									 %instead of title
}

\lstdefinestyle{ansic}{
	belowcaptionskip=1\baselineskip,
	frame=r,
	xleftmargin=\parindent,
	language=C,
	commentstyle=\color{green}\ttfamily,
	keywordstyle=\color{blue}\ttfamily,
	basicstyle=\footnotesize\ttfamily
}

\lstdefinestyle{asm}{
	belowcaptionskip=1\baselineskip,
	frame=r,
	xleftmargin=\parindent,
	language=[x86masm]Assembler,
	basicstyle=\footnotesize\ttfamily,
	commentstyle=\itshape\color{red},
}

\lstdefinestyle{cmds}{
	belowcaptionskip=1\baselineskip,
	frame=l,
	xleftmargin=\parindent,
	language=bash,
	basicstyle=\footnotesize\ttfamily,
	commentstyle=\itshape\color{red},
}

\lstdefinestyle{py}{
	belowcaptionskip=1\baselineskip,
	frame=l,
	xleftmargin=\parindent,
	language=python,
	basicstyle=\footnotesize\ttfamily,
	commentstyle=\itshape\color{green},
}

% End of listings setup
%-------------------------------------------------------------------------------
\pagestyle{fancyplain} % Makes all pages in the document conform to 
					   %the custom headers and footers
\fancyhead{} % No page header - if you want one, create it in the same 
             %way as the footers below
\fancyfoot[L]{} % Empty left footer
\fancyfoot[C]{} % Empty center footer
\fancyfoot[R]{\thepage} % Page numbering for right footer
\renewcommand{\headrulewidth}{0pt} % Remove header underlines
\renewcommand{\footrulewidth}{0pt} % Remove footer underlines
\setlength{\headheight}{13.6pt} % Customize the height of the header
\usepackage[margin=3cm]{geometry}

\numberwithin{equation}{section} % Number equations within sections (i.e. 1.1, 1.2, 2.1, 2.2 instead of 1, 2, 3, 4)
\numberwithin{figure}{section} % Number figures within sections (i.e. 1.1, 1.2, 2.1, 2.2 instead of 1, 2, 3, 4)
\numberwithin{table}{section} % Number tables within sections (i.e. 1.1, 1.2, 2.1, 2.2 instead of 1, 2, 3, 4)

\setlength\parindent{0pt} % Removes all indentation from paragraphs - comment this line for an assignment with lots of text
%-------------------------------------------------------------------------------
% GLOSSARY AND ACRONYMS ENTRIES
%-------------------------------------------------------------------------------

\newglossaryentry{BA}{name={Binary Analysis},description={Analysis of binary 
		computer programs, binaries in short, and the code within them to truly 
		understand their actual behaviour as opposed to their intended 
		behaviour}}
\newglossaryentry{SA}{name={Static Analysis},description={Analysis of a binary 
without running it}}
\newglossaryentry{DA}{name={Dynamic Analysis},description={Analysis of a binary 
at runtime}}

\newacronym{OS}{OS}{Operating System}
\newacronym{API}{API}{Application Programming Interface}
\newacronym{ISA}{ISA}{Instruction Set Architecture}
\newacronym{FS}{FS}{File System}
\newacronym{IO}{I/O}{Input/Output}
\newacronym{VM}{VM}{Virtual Machine}
\newacronym{VMM}{VMM}{Virtual Machine Monitor}
\newacronym{370-XA}{370-XA}{System 370 Extended Architecture}
\newacronym{VMF}{VMF}{Virtual Machine Facility}
\newacronym{VMA}{VMA}{Virtual Machine Assist}
\newacronym{ECPS}{ECPS}{Extended Control Program Support}
\newacronym{STB}{STB}{Shadow Table Bypass}
\newacronym{IA-32}{IA-32}{Intel Architecture 32 Bit}
\newacronym{IA-64}{IA-64}{Intel Architecture 64 Bit}
\newacronym{ABI}{ABI}{Application Binary Interface}
\newacronym{ELF}{ELF}{Executable and Linkable Format}
\newacronym{PE}{PE}{Portable Executable}
\newacronym{PLT}{PLT}{Procedure Linkage Table}
\newacronym{GDT}{GDT}{Global Description Table}
\newacronym{LDT}{LDT}{Local Description Table}
\newacronym{PPN}{PPN}{Physical Page Number}
\newacronym{TLS}{TLS}{Thread Level Storage}
%-------------------------------------------------------------------------------
%	TITLE SECTION
%-------------------------------------------------------------------------------

\newcommand{\horrule}[1]{\rule{\linewidth}{#1}} % Create horizontal rule command with 1 argument of height

% create new commands and environments for code excerpts
\newcommand{\addr}[1]{{\fontfamily{pcr}\selectfont{0x00#1}}}
\newcommand{\reg}[1]{{\fontfamily{pcr}\selectfont{#1}}}
\newcommand{\cmp}[2]{{\fontfamily{pcr}\selectfont{\textbf{cmp} #1, #2}}}
\newcommand{\mov}[2]{{\fontfamily{pcr}\selectfont{\textbf{mov} #1, #2}}}
\newcommand{\movsx}[2]{{\fontfamily{pcr}\selectfont{\textbf{movsx} #1, #2}}}
\newcommand{\lea}[2]{{\fontfamily{pcr}\selectfont{\textbf{lea} \reg{#1}, [#2]}}}
\newcommand{\radare}[1]{{\fontfamily{pcr}\selectfont\textbf{#1}}}
\newcommand{\main}{{\fontfamily{pcr}\selectfont\textbf{main()}}}
\newcommand{\func}[2]{{\fontfamily{pcr}\selectfont\textbf{#1(#2)}}}
\renewcommand{\floatpagefraction}{.8}%

\title{	
\normalfont \normalsize 
\textsc{University of Genova, DIBRIS} \\ [25pt]
\horrule{0.5pt} \\[0.4cm] % Thin top horizontal rule
\huge Binary Analysis and Secure Coding \\
\horrule{2pt} \\[0.5cm] % Thick bottom horizontal rule
}

\author{Andrea Basso} % Your name

\date{\normalsize\today} % Today's date or a custom date
\makeglossaries
\begin{document}

\maketitle % Print the title
\tableofcontents
\chapter{Binary Analysis}
Binary analysis can be carried out with two different techniques:
\begin{itemize}
	\item Static Analysis:
		\subitem \textcolor{green}{The entire binary can be potentially 
		analysed in one go};
		\subitem \textcolor{green}{No need for an architecture capable of 
		running the binary};
		\subitem \textcolor{red}{No knowledge of the runtime states};
	\item Dynamic Analysis:
		\subitem \textcolor{green}{Often simpler};
		\subitem \textcolor{green}{Possibility to observe a particular 
		execution};
		\subitem \textcolor{red}{Chance to miss interesting parts of the code};
		\subitem \textcolor{red}{Potentially dangerous if analysisng malware};
\end{itemize}
It can present several challenges as:
\begin{itemize}
	\item it is in a low level language - i.e. Assembly - that can present 
	itself in different flavours (AT\&T, Intel, non x86 platforms);
	\item There might be no symbols (e.g. stripped binaries);
	\item Code and data are mixed together;
	\item Impossible to remove/add instructions without "bricking" the program.
\end{itemize}

\section{Execution}
Unless one is programming the bare metal, programs are generally executed on an 
Operating System (OS hereafter) which is an \textbf{abstract machine}.
A running profram is a process, which has its own resources and can use a 
subset of the ISA (implemented on the actual CPU or emulates), an abstract 
model of a CPU and the API of the OS (e.g. syscall).
\subsection{Emulation, Virtualisation, Containers}
As computers became more accessible to people the idea of simply time sharing 
the machine became unfeasible; system had to be made reduntant and fault 
tolerant whilst being able to be shared across the entire userbase. Having 
multiple systems is beneficial for:
\begin{itemize}
	\item \textbf{Isolation} Systems must be isolated for stability reasons in 
	such a way that programs do not interfere with each other for stability 
	(i.e. buggy programs shall not cause disruption to other processes);
	\item  \textbf{Performance} Placing an application on it's own system 
	allows it to have exclusive access to the system's resources. User level 
	segregation does not effectively isolate applications as scheduling 
	priority, memory demand, network and FS I/O can generate interference 
	between programs.
\end{itemize}
However having multiple systems is not always the most fruitful and effective 
solution and in the 60's IBM started working on the first \textit{virtual 
machines} where it was crucial to be able to time-share expensive main frames. 
A VM is a fully isolated copy of the underlying hardware that enable 
applications to run as they were on "bare metal".
\subsubsection{Virtual Machine Monitor}
The VMM is the software component that hosts the virtual machines; the
VMM is generally called \textit{host} and the VM's are called \textit{guests}. 
It's duties are, among the others, the provision of virtulised CPU, virtualised 
resources such as I/O devices, storage, memory and isolation between VM's.
\subsubsection{Internet Services Virtualisation}
In order to face the ever more demanding Internet applications a new kind of 
system was adopted: the multi-tier system architecture. THis common way of 
building Internet applications entails the separation of the web and database 
server and easily enables load-balancing and clustering. The system has the 
advantage of being easier to manage and extremely fault tolerant, but this 
requires the additional cost of additional hardware purchases and management. 
In this scenario the system virtualisation provides the advantages of 
componitisation without the need for additional hardware (and exacerbated by 
Zipf's law\footnote{The frequency of an event is proportional to 
$x^{-\alpha}$ where $x$ is the rank of the event in comparison to other 
events}). Studies have found that the internet services behave according to a 
Zipfian distribution, i.e. if we observe the web caches we see that only a few 
web services are always active and used the most of the others just lay 
dormant; thus isolation is no longer a worthy option unless services are 
consolidated to a single machine and virtualisation provides the best benefit 
for the cost as it avoids the waste of computing power and reduces the 
maintenance costs.
\subsubsection{Requirements for VM's}
In 1974 Popek and Goldberg defined what they believed where the formal 
requirements for a virtualisable computer architecture: \textit{"For any 
computer a virtual machine monitor can be constructed if the set of 
instructions is a subset of the privileged instructions"}; the most essential 
requirement a computer architecture must have in order to be virtualisable is 
that privileged instructions must launch a trap and render control to the VMM 
so it can decide whether the instruction can be executed or not. The three 
essential characteristics of VM's are:

\begin{enumerate}
	\item Any program run under the VMM should exhibit an effect identical with 
	that demonstrated if the program had run on the original machine (with the 
	exception of the timing effects related to the virtualisation overhead);
	\item A statistically dominant subset of the virtual processor's 
	instructions are directly executed by the real processor - i.e. the virtual 
	machine occasionally relinquishes the real processor to the virtual 
	processor;
	\item The VMM is in complete control of system resources and the virtual 
	machine does not have direct access to the system's real resources.
\end{enumerate}
\subsection{Full System Virtualisation}
Full sytem virtualisation provides a virtual replica of the system's hardware 
so that the OS's and software may run on the VM exactly as it would on "bare 
metal".
\paragraph{370-XA Virtualisation} The VM/370 is a VMM that was first delivered 
in the VMF/370 and ran the 370-XA, an architecture designed to maximise the 
virtualisation performance. The platform provided \textit{"assists"} in the 
architecture to boost the virtualisation performance of certain repetitive 
operations. The VMM had to simulate the instruction beforehand to ensure 
safety; the assist development allowed the execution of the simulated 
instructions on the hardware. Other assists supplanted the frequently executed 
instructions that still required VMM intervention and were enabled by setting 
them in \textit{interpretive execution} mode. Assists included VMA, ECPS, STB 
and many others (IBM developed more than 100 in total).
\paragraph{IA-32} It was the most dominating architecture (now supplanted by 
x64) but it was never meant to be virtualised although that would have had 
enormous benefits; the reason behind this resides in the fact that there is a 
limited subset of instructions that does not need higher privilege to be run 
and cripple a machine. This is not a problem in systems with single OS, but it 
becomes difficult to manage when multiple OS's are running at once. Furthermore 
the huge amount of different devices and drivers for the IA-32 made 
virtualisation a big deal. However IA-32 has been finally virtualised according 
to the following
\begin{enumerate}
	\item Non-sensitive, non-privileged instructions may be run directly on the 
	processor;
	\item Sensitive, Privileged instructions trap - i.e. the processor sends an 
	interrupt and the VMM instercepts it for proper management of the 
	instruction
	\item Sensitive, but not privileged instructions are detected and the VMM 
	ensures they never get executed as they cannot be handled with a trap.
\end{enumerate}
\paragraph{IA-64} It presents more or less the same drawbacks of the IA-32, but 
it provides an important feature as it allows to run VM's in a higher ring of 
the VMM and traps from the higher rings can be intercepted by lower rings.
\subsubsection{Drawbacks of full system virtualisation}
The advantage of full system virtualisation is that OS's and programs can run 
unmodified and completely oblivious of the environment in which they are 
actually running. However there are drawbacks related to the fact that IA-32 
and IA-64 were not created to be virtualised and thus a lot of effort must be 
put into efficiently virtualising hardware and memory management. When an 
application demands for a memory page the system translates the virtual address 
to a physiscal one thanks to page table. Unused pages can be written to disk 
when inactive or when more memory is required. The VMM must intercept all of 
the memory calls and translate VM space into the system's real space using 
another pag table, retrieve the memory and return it to the VM.
\subsection{Paravirtualisation} Paravirtualisation tries to mitigate the 
performance hit by modifying the OS so that tasks are performed by the VMM 
directly and not by the CPU. There are several commecrcial and open solutions 
that can efficiently carry out paravirtualisation like:
\begin{itemize}
	\item Xen Project
	\item Citrix Xen Server
	\item Denali
	\item VMWare Sphere
\end{itemize}
The VMM provides the VM's with a software interface that is similar yet not 
entirely identical to the underlying hardware. Paravirtualisation provides 
specifically defined hooks by which it tries to minimise the number of 
operations that must be carried out in the virtual environment. The guest OS 
shall be modified for the paravirtualisation API, although this requirement is 
less strict nowadays as more and more paravirtualsation enabled components can 
be found under the GPL and other free licenses.
\subsection{Other virtualisation techniques}
There are other virtualisation techniques like:
\begin{itemize}
	\item Emulation: replication of a specific architecture that enables a 
	machine to behave like another
	\item Containerisation: OS-level virtualisation that allows for the 
	presence of isolated user-space instances called containers.
\end{itemize}
\section{Binaries}
Executables are created in compliance with a specific ABI that depends on the 
architecture. The ABI establishes:
\begin{itemize}
	\item The executable and object file formats
	\item fundamental types (size alignment for int, long and other primitives)
	\item How data types are laid out in memory
	\item Function and system call conventions
	\item How programs are started up (initialisation, dynamic linking, etc.)
\end{itemize}
\subsection{Executable and Linkable Format}
It is a flexible format that can be used for:
\begin{itemize}
	\item executables;
	\item dynamic libraries (*.so);
	\item object files (also called relocatable files).
\end{itemize}
The ELF format is a sequence of headers, segments and sections containing the 
necessary information and data for the execution and linkage of the executable.
\subsubsection{File Header}
The file header is located at the beginning of the file and is used to locate 
the other parts.\newpage
\begin{lstlisting}[style=ansic, caption={File Header}, label=ehdr]
	typedef struct {
		unsigned char e_ident;
		Elf64_Half 		e_type;
		Elf64_Half 		e_machine;
		Elf64_Word 		e_version;
		Elf64_Addr 		e_entry;
		Elf64_Off			e_phoff;
		Elf64_Off			e_shoff;
		Elf64_Word		e_flags;
		Elf64_Half 		e_ehsize;
		Elf64_Half 		e_phentsize;
		Elf64_Half 		e_phnum;
		Elf64_Half 		e_shentsize;
		Elf64_Half 		e_shnum;
		Elf64_Half 		e_shstrndx;
	} Elf64_Ehdr;
\end{lstlisting}
where:
\begin{itemize}
	\item {\ttfamily e\_ident} is an array of bytes that identifies the file as 
	an ELF and provides info about the data representation of the object file 
	structures. The bytes of the identifier are:
	\subitem {\ttfamily e\_ident[EI\_MAG0], e\_ident[EI\_MAG1], 
	e\_ident[EI\_MAG2], e\_ident[EI\_MAG3]} contain the so called magic number 
	'0x7fELF';
	\subitem {\ttfamily e\_ident[EI\_CLASS]} identifies the class of the ELF 
	(32bit or 64bit);
	\subitem {\ttfamily e\_ident[EI\_DATA]} specifies the encoding of the 
	object file data structures (little or big endian);
	\subitem {\ttfamily e\_ident[EI\_VERSION]} identifies the version of the 
	object file format and is presently set to EV\_CURRENT which has value 1;
	\subitem {\ttfamily e\_ident[EI\_OSABI]} identifies the OS and ABI version 
	for which the object file is prepared. Some fields in other ELF structures 
	have flags that can be interpreted based on the value of this field
	\subitem {\ttfamily e\_ident[EI\_ABIVERSION]} Identifies te version of the 
	ABI fdor which the binary is prepared. The interpretation of this field 
	depends on the value of the {\ttfamily e\-ident[EI\_OSABI]}
	\item {\ttfamily e\_type} Identifies the object file type and its values 
	can be:\newline
	\begin{center}
			\begin{tabular}{|c|c|c|}
			\hline
			\textbf{Name} & \textbf{Value} & \textbf{Purpose} \\ \hline
			{\ttfamily ET\_NONE} & 0x0000 & No file type \\ \hline
			{\ttfamily ET\_REL} & 0x0001 & Relocatable file type \\ \hline
			{\ttfamily ET\_EXEC} & 0x0002 & Executabler \\ \hline
			{\ttfamily ET\_DYN} & 0x0003 & Shared Object \\ \hline
			{\ttfamily ET\_CORE} & 0x0004 & Core File \\ \hline
			{\ttfamily ET\_LOOS} & 0xFE00 & Environment specific use \\ \hline
			{\ttfamily ET\_HIOS} & 0xFEFF & Environment specific use\\ \hline
			{\ttfamily ET\_LOPROC} & 0xFF00 & Processor specific use\\ \hline
			{\ttfamily ET\_HIPROC} & 0xFFFF & Processor specific use\\ \hline
		\end{tabular}
	\end{center}
	\item {\ttfamily e\_machine} identifies the architecture
	\item {\ttfamily e\_version} identifies the version of the current file 
	format
	\item {\ttfamily e\_entry} contains the virtual address of the entry point
	\item {\ttfamily e\_phoff} contains the file offset in bytes of the program 
	header table
	\item {\ttfamily e\_shoff} contains the offset in bytes of the section 
	header table
	\item {\ttfamily e\_flags} contains processor specific flags
	\item {\ttfamily e\_ehsize} contains the size in bytes of the ELF header
	\item {\ttfamily e\_phentsize} contains the size in bytes of a progam 
	header entry
	\item {\ttfamily e\_phnum} contains the number pf entries in the program 
	header table
	\item {\ttfamily e\_shentsize} contains the size in bytes of a section 
	headeer table entry
	\item {\ttfamily e\_shnum} contains the number of entries in the section 
	header table
	\item {\ttfamily e\_shstrndx} section header table index of the string table
\end{itemize}
\subsubsection{Sections}
The code and data in an ELF binary are logically divided into contiguous 
nonoverlapping chunks called sections; their structure is not predetrmined and 
varies depending on the content.
Sections contain all the information in an ELF file, with the exception of the 
file header and can be identified by their index into the section header table. 
Strictly speaking the organisation into sections is intended to be convenient 
at linking; hence not all of the sections are actually needed while setting up 
a process and the virtual memory (e.g. symbols and relocation sections do not 
contain info needed during execution). As sections are only there for the 
linker the presence of the section header table is not mandatory (and the 
{\ttfamily e\_shoff} is set to 0 in this case).
\paragraph{Section Indices}
Section indices 0, 0xFF00-0xFFFF are reserved for special purposes as per the 
following:
\begin{center}
	\begin{tabular}{|c|c|p{6cm}|}
		\hline
		\textbf{Name} 	& \textbf{Value} 	& \textbf{Description}\\ \hline
		SHN\_UNDEF & 0 & Used to mark undefined or meaningless section 
		reference\\ \hline
		SHN\_LOPROC & 0xFF00 & Processor-specific use\\ \hline
		SHN\_HIPROC & 0XFF1F & Processor-specific use\\ \hline
		SHN\_LOOS & 0xFF20 & Environment-specific use\\ \hline
		SHN\_HIOS & 0xFF3F & Environment-specific use\\ \hline
		SHN\_ABS & 0xFFF1 & Indicates that the reference is an absolute value\\ 
		\hline
		SHN\_COMMON & 0xFFF2 & Indicates a symbol that has been declared as a 
		common block (tentative declaration in C)\\ \hline
	\end{tabular}
\end{center}
\textcolor{red}{The first entry in the section header table (at index 0) 
\textbf{MUST} contain all zeroes.}
\paragraph{Section Header Entries}
The structure of a section header is as per the following
\begin{lstlisting}[style=ansic, caption={Section Header}, label=shdr]
typedef  struct {
	Elf64_Word sh_name;
	Elf64_Word sh_type;
	Elf64_Xword sh_flags;
	Elf64_Addr sh_addr;
	Elf64_Off sh_offset;
	Elf64_Xword sh_size;
	Elf64_Word sh_link;
	Elf64_Word sh_info;
	Elf64_Xword sh_addralign;
	Elf64_Xword sh_entsize;
} Elf64_Shdr;
\end{lstlisting}
where:
\begin{itemize}
	\item {\ttfamily sh\_name} contains the offset, in bytes, to the section 
	name relatively to the start of the section name string table;
	\item {\ttfamily sh\_type} identifies the section type detailed as follows
	\begin{center}
		\begin{tabular}{|c|c|p{6cm}|}
			\hline
			\textbf{Name} & \textbf{Value} & \textbf{Meaning} \\ \hline
			{\ttfamily SHT\_NULL} & 0 & Marks an unused section 
			header \\ \hline
			{\ttfamily SHT\_PROGBITS} & 1 & Contains program data such as 
			machine instructions and constants \\ \hline
			{\ttfamily SHT\_SYMTAB} & 2 & Contains the static linking symbol 
			table \\ \hline
			{\ttfamily SHT\_STRTAB} & 3 & Contains a string table \\ 
			\hline
			{\ttfamily SHT\_RELA} & 4 & Contains "Rela" type 
			relocation entries \\ \hline
			{\ttfamily SHT\_HASH} & 5 & Contains a symbol hash table \\ \hline
			{\ttfamily SHT\_DYNAMIC} & 6 & Contains dynamic linking tables\\ 
			\hline
			{\ttfamily SHT\_NOTE} & 7 & Contains note information\\ \hline
			{\ttfamily SHT\_NOBITS} & 8 & Contains uninitialised 
			space; does not occupy any space in the file\\ \hline
			{\ttfamily SHT\_REL} & 9 & Contains "Rel" type relocation entries\\ 
			\hline
			{\ttfamily SHT\_SHLIB} & 10 & Reserved\\ \hline
			{\ttfamily SHT\_DYNSYM} & 11 & Contains a dynamic loader symbol 
			table\\ 
			\hline
			{\ttfamily SHT\_LOOS} & 0x6000 0000 & Environment-specific use\\ 
			\hline
			{\ttfamily SHT\_HIOS} & 0x6FFF FFFF & Environment-specific use \\ 
			\hline
			{\ttfamily SHT\_LOPROC} & 0x7000 0000 & Processor-specific use\\ 
			\hline
			{\ttfamily SHT\_HIPROC} & 0x7FFF FFFF & Processor-specific use\\ 
			\hline
		\end{tabular}
	\end{center}
	\item {\ttfamily sh\_flags} identifies the attributes of the section name 
	as listed:
	\begin{center}
		\begin{tabular}{|c|c|p{6cm}|}
			\hline
			\textbf{Name} & \textbf{Value} & \textbf{Meaning} \\ \hline
			{\ttfamily SHF\_WRITE} & 0x1 & Contains writeable data\\ 
			\hline
			{\ttfamily SHF\_ALLOC} & 0x2 & Section allocated in memory image of 
			a program\\ 
			\hline
			{\ttfamily SHF\_EXECINSTR} & 0x4 & Contains executable 
			instructions\\ 
			\hline
			{\ttfamily SHF\_MASKOS} & 0x0F00 0000 & Environment-specific use\\ 
			\hline
			{\ttfamily SHF\_MASKPROC} & 0xF000 0000 & Processor-specific use\\ 
			\hline
		\end{tabular}
	\end{center}
	\item {\ttfamily sh\_addr} contains the virtual address of the beginning of 
	the section in memory. If the section is not allocated in memory this shall 
	be 0;
	\item {\ttfamily sh\_offset} contains the offset of the section in bytes 
	from the beginning of the file;
	\item {\ttfamily sh\_size} contains the size of the section (unless it's a 
	{\ttfamily SHT\_NOBITS});
	\item {\ttfamily sh\_link} contains the index of an associated section. 
	This field is used for several purposes as detailed in the following:
	\begin{center}
		\begin{tabular}{|c|c|}
			\hline
			\textbf{Type} & \textbf{Associated Section} \\ \hline
			{\ttfamily SHT\_DYNAMIC} & String table used by entries in this 
			section\\ \hline
			{\ttfamily SHT\_HASH} & Symbol table in which the hash table 
			applies\\ \hline
			{\ttfamily SHT\_REL/RELA} & Symbol table referenced by the 
			relocations\\ \hline
			{\ttfamily SHT\_SYMTAB/DYNSYM} & String table used in entries used 
			in these sections\\ \hline
		\end{tabular}
	\end{center}
	\item {\ttfamily sh\_info} contains extra info about the section as 
	explained below:
	\begin{center}
		\begin{tabular}{|c|c|}
			\hline
			\textbf{Type} & \textbf{Info} \\ \hline
			{\ttfamily SHT\_REL/RELA} & Index of the section which the 
			relocation applies to\\ \hline
			{\ttfamily SHT\_SYMTAB/DYNSYM} & Index of the first non-local 
			symbol (i.e. number of local symbols)\\ \hline
		\end{tabular}
	\end{center}
	\item {\ttfamily sh\_addralign} contains the required alignment of the 
	section and must be a power of 2;
	\item {\ttfamily sh\_entsize} contains the size in bytes, of each entry, 
	for sections that contain fixed-size entries (zero otherwise).
\end{itemize}
The standard sections for Code/Data and other object file are reported in 
tables \ref{codedata} and \ref{objfile}.
\begin{table}[!htbp]
	\begin{center}
		\begin{tabular}{|c|c|c|c|}
			\hline
			\textbf{Name} & \textbf{Type} & \textbf{Flags} & \textbf{Use}\\ 
			\hline
			{\ttfamily .bss} & {\ttfamily SHT\_NOBITS} & {\ttfamily 
			SHF\_ALLOC/WRITE} & Uninitialised data\\ \hline
			{\ttfamily .data} & {\ttfamily SHT\_PROGBITS} & {\ttfamily 
			SHF\_ALLOC/WRITE} & Initialised data\\ \hline
			{\ttfamily .interp} & {\ttfamily SHT\_PROGBITS} & {\ttfamily 
			SHF\_ALLOC} & Program interpreter path name\\ \hline
			{\ttfamily .rodata} & {\ttfamily SHT\_PROGBITS} & {\ttfamily 
			SHF\_ALLOC} & Read only data (const and literals)\\ \hline
			{\ttfamily .text} & {\ttfamily SHT\_PROGBITS} & {\ttfamily 
			SHF\_ALLOC/EXECINSTR} & Executable code\\ \hline
		\end{tabular}
		\caption{Code and Data Standard Sections}
		\label{codedata}
	\end{center}
\end{table}
\begin{table}[!htbp]
	\begin{center}
		\begin{tabular}{|c|c|c|c|}
			\hline
			\textbf{Name} & \textbf{Type} & \textbf{Flags} & \textbf{Use}\\ 
			\hline
			{\ttfamily .comment} & {\ttfamily SHT\_PROGBITS} & {\ttfamily 
				None} & Version Control Info\\ \hline
			{\ttfamily .dynamic} & {\ttfamily SHT\_DYNAMIC} & {\ttfamily 
				SHF\_ALLOC/WRITE} & Dynamic Linking Tables\\ \hline
			{\ttfamily .dynstr} & {\ttfamily SHT\_STRTABS} & {\ttfamily 
				SHF\_ALLOC} & String table for the {\ttfamily .dynamic} 
				segment\\ \hline
			{\ttfamily .dynsym} & {\ttfamily SHT\_DYNSYM} & {\ttfamily 
				SHF\_ALLOC} & Read only data (const and literals)\\ \hline
			{\ttfamily .got} & {\ttfamily SHT\_PROGBITS} & {\ttfamily 
				machine dependent} & Global Offset Table\\ \hline
			{\ttfamily .hash} & {\ttfamily SHT\_HASH} & {\ttfamily 
				SHT\_ALLOC} & Symbol Hash Table\\ \hline
			{\ttfamily .note} & {\ttfamily SHT\_NOTE} & {\ttfamily 
				none} & Note Section\\ \hline
			{\ttfamily .plt} & {\ttfamily SHT\_PROGBITS} & {\ttfamily 
				machine dependent} & Procedure Linkage Table\\ \hline
			{\ttfamily .rel[\textit{name}]} & {\ttfamily SHT\_REL} & 
			{\ttfamily SHT\_ALLOC} & Relocations for section [\textit{name}]\\ 
			\hline
			{\ttfamily .rela[\textit{name}]} & {\ttfamily SHT\_RELA} & 
			{\ttfamily SHT\_ALLOC} & As Above \\ 
			\hline
			{\ttfamily .shstrtab} & {\ttfamily SHT\_STRTAB} & {\ttfamily None} 
			& Section Name String Table\\ \hline
			{\ttfamily .strtab} & {\ttfamily SHT\_STRTAB} & {\ttfamily None} & 
			String Table\\ \hline
			{\ttfamily .symtab} & {\ttfamily SHT\_SYMTAB} & {\ttfamily 
			SHT\_ALLOC} & Linker Symbol Table\\ \hline
		\end{tabular}
		\caption{Other Object Files Standard Sections}
		\label{objfile}
	\end{center}
\end{table}

\subsection{String Table}
Contains the strings used for section and symbol names. It is an array of null 
terminated strings. Section header table entries, and symbol table entries 
refer to strings in a string table with an index relative to the beginning of 
the string table. \textcolor{red}{The first byte of the table must be null so 
that the index 0 always refers to a non-existent name.}

\subsection{Symbol Table}
The first symbol table entry is reserved and must be all zeroes ({\ttfamily 
STN\_UNDEF} is useed as a reference to this entry).
The structure of a symbol table entry is

\begin{lstlisting}[style=ansic, caption={Symbol Table Entry}, label=shdr]
	typedef  struct {
		Elf64_Word sh_name;
		unsigned char 	st_info;
		unsigned char 	st_other;
		Elf64_Half 		st_shndx;
		Elf64_Addr 		st_value;
		Elf64_Xword 	st_size;
	} Elf64_Sym;
\end{lstlisting}

where:
\begin{itemize}
	\item {\ttfamily st\_name} contains the offset in bytes to the symbol name 
	from the beginning of the string table;
	\item {\ttfamily st\_info} contains the symbol type and its binding 
	attributes - i.e. the scope. The binding attributes are contained in the 
	high-order 4 bits of the eight bit byte and the symbol type is contained in 
	the lower order 4 bits. The availble bindings and types are reported in 
	tables \ref{bindings} and \ref{types}.
	\begin{table}[!htbp]
		\begin{center}
			\begin{tabular}{|c|c|c|c|}
				\hline
				\textbf{Name} & \textbf{Value} & \textbf{Description}\\ 
				\hline
				{\ttfamily STB\_LOCAL} & 0 & Not visible out of the object 
				files\\ \hline
				{\ttfamily STB\_GLOBAL} & 1 & Visible to all object files\\ 
				\hline
				{\ttfamily STB\_WEAK} & 2 & Global but with lower precedence\\ 
				\hline
				{\ttfamily STB\_LOOS} & 10 & Env specific use\\ \hline
				{\ttfamily STB\_HIOS} & 12 & Env specific use\\ \hline
				{\ttfamily STB\_LOPROC} & 13 & Processor specific use\\ \hline
				{\ttfamily STB\_HIPROC} & 15 & Processor specific use\\ \hline
			\end{tabular}
			\caption{Symbol Bindings}
			\label{bindings}
		\end{center}
	\end{table}
	\begin{table}[!htbp]
		\begin{center}
			\begin{tabular}{|c|c|c|c|}
				\hline
				\textbf{Name} & \textbf{Value} & \textbf{Description}\\ 
				\hline
				{\ttfamily STT\_NOTYPE} & 0 & No type\\ \hline
				{\ttfamily STT\_OBJECT} & 1 & Data Object\\ \hline
				{\ttfamily STT\_FUNC} & 2 & Function entry point\\ \hline
				{\ttfamily STT\_SECTION} & 3 & Symbol is associated with a 
				section\\ \hline
				{\ttfamily STT\_FILE} & 4 & Source file associated with the 
				object file\\ \hline
				{\ttfamily STT\_LOOS} & 10 & Env specific use\\ \hline
				{\ttfamily STT\_HIOS} & 12 & Env specific use\\ \hline
				{\ttfamily STT\_LOPROC} & 13 & Processor specific use\\ \hline
				{\ttfamily STT\_HIPROC} & 15 & Processor specific use\\ \hline
			\end{tabular}
			\caption{Symbol Types}
			\label{types}
		\end{center}
	\end{table}
	\item {\ttfamily st\_other} is reserved for future use and must be 0;
	\item {\ttfamily st\_shndx} contains the index of the section the index is 
	defined into. For undefined symbols it is {\ttfamily SHN\_UNDEF}, for 
	absolute symbols it is {\ttfamily SHN\_ABS} and for common ones is 
	{\ttfamily SHN\_COMMON};
	\item {\ttfamily st\_value} contains the value of the symbol. it may be 
	aboslute or relocatable address. In relocatable files this field contains 
	the alignment constraint for common symbols and a section relative offset 
	for defined relocatable symbol. In executables and standard object files 
	this contains a virtual address for the defined relocatable symbol.
	\item {\ttfamily st\_size} contains the size associated with the symbol. If 
	a symbol does not have an associated size this field contains 0.
\end{itemize}
\subsection{Relocations}
These sections contain information used by the linker to perform relocations; 
essentially these are tables with each entry detailing a particular address at 
whic the relocation must be performed along with the instructions to resolve 
the value that needs to be plugged at that value. Of course static relocations 
are resolved at compile time and only dynamic will remain. 
These sections can be {\ttfamily .rel.*} or {\ttfamily .rela.*} and their type 
defines the symbol resolution method.
\paragraph{.rel.*}
The {\ttfamily .rel.*} sections contain the addend part of the relocation from 
the original part and are smaller than {\ttfamily .rela.*}. The underlying 
data structure is reported in \ref{rel}.
\begin{lstlisting}[style=ansic, caption={.rel.* entries}, label=rel]
typedef  struct {
	Elf64_Addr r_offset; // Address of reference
	Elf64_Xword r_info;  // Symbol index and type of relocation
} Elf64_Rel;
\end{lstlisting}
\paragraph{.rela.*}
This is larger than the former and provides an explicit field for the full 
addend.
\begin{lstlisting}[style=ansic, caption={.rel.* entries}, label=rel]
typedef  struct {
	Elf64_Addr r_offset;   // Address of reference
	Elf64_Xword r_info;    // Symbol index and type of relocation
	Elf64_Sxword r_addend; // Constant part of the expression
} Elf64_Rela;
\end{lstlisting}
where:
\begin{itemize}
	\item {\ttfamily r\_offset} indicates the location at which the relocation 
	must be applied. For a relocatable file this is the offset, in btes, from 
	the beginning of the section to the beginning of the storage unit being 
	relocated. For an executable or shared object, this is the virtual address 
	of the storage unit being relocated.
	\item {\ttfamily r\_info} contains both a symbol table index and a 
	relocation type. The symbol table index identifies the symbol which value 
	should be used in the relocation.
	The type instead qualifies the method used for the relocation; these can 
	vary wildly in the real world, but the most important are {\ttfamily 
	R\_X86\_64\_GLOB\_DAT} and {\ttfamily R\_X86\_64\_JUMP\_SLO}. 
	\subitem The first one is used to compute the address of a data symbol and 
	plug it in the correct offset into the {\ttfamily .got} section.
	\subitem The latter is called jump slot and they have their offset in the 
	{\ttfamily .got.plt} (the stubs in the PLT use them to compute the jump 
	target as an offset from the {\ttfamily rip} register).
	They can be identified by application of the macro in \ref{elf64relmacros}.
	\begin{lstlisting}[style=ansic,caption={Macros},label=elf64relmacros]
	#define ELF64_R_SYM(i)((i) >> 32)
	#define ELF64_R_SYM(i)((i) & 0xffffffff)
	#define ELF64_R_INFO(s, t)(((s) << 32) + ((t) & 0xffffffff))
	\end{lstlisting}
	\item {\ttfamily r\_addend} specifies a constant addend used to compute the 
	value to be stored in the relocated field.
\end{itemize}
\pagebreak
\subsection{Program Header Table}
The program header table provides the segment view as opposed to the section 
perspective detailed above and has a different purpose indeed. In fact sections 
are exclusively relevant to the static linker, while segments are used by the 
dynamic linker when loading an ELF into a process for execution to locate the 
relevant code and data and decide what to load in virtual memory. A single 
segment may contain multiple sections. The underlying data structure can be 
found in listing \ref{elf64phdr}
\begin{lstlisting}[style=ansic, caption={Program Header}, label=elf64phdr]
	typedef  struct {
		Elf64_Word 	p_type;    // Segment type
		Elf64_Word 	p_flags;   // Segment attributes
		Elf64_Off 	p_offset;  // Constant part of the expression
		Elf64_Addr 	p_vaddr;   // Virtual address in memory
		Elf64_Addr 	p_paddr;   // Reserved
		Elf64_Xword p_filesz;  // Size of the segment in the file
		Elf64_Xword p_memsz;   // Size of the segment in memory
		Elf64_Xword p_align;   // Alignment of segment
	} Elf64_Rela;
\end{lstlisting}
where:
\begin{itemize}
	\item {\ttfamily p\_type} indicates the type of the segment and can have 
	the following values:
	\begin{table}[!htbp]
		\begin{center}
			\begin{tabular}{|c|c|c|}
				\hline
				\textbf{Name} & \textbf{Value} & \textbf{Description}\\ 
				\hline
				{\ttfamily PT\_NULL} & 0 & Unused entry\\ \hline
				{\ttfamily PT\_LOAD} & 1 & Loadable segment\\ \hline
				{\ttfamily PT\_DYNAMIC} & 2 & Dynamic linking tables\\ \hline
				{\ttfamily PT\_INTERP} & 3 & Program interpreter path\\ \hline
				{\ttfamily PT\_NOTE} & 4 & Note sections\\ \hline
				{\ttfamily PT\_SHLIB} & 5 & Reserved\\ \hline
				{\ttfamily PT\_PHDR} & 6 & Program header table\\ \hline
				{\ttfamily PT\_LOOS} & 0x60000000 & Environment specific use\\ 
				\hline
				{\ttfamily PT\_HIOS} & 0x6FFFFFFF & Environment specific use\\ 
				\hline
				{\ttfamily PT\_LOPROC} & 0x70000000 & Processor specific use\\ 
				\hline
				{\ttfamily PT\_HIPROC} & 0x7FFFFFFF & Processor specific use\\ 
				\hline
			\end{tabular}
			\caption{Segment types}
			\label{segtypes}
		\end{center}
	\end{table}
	The fields {\ttfamily PT\_LOAD}, {\ttfamily PT\_DYNAMIC} and {\ttfamily 
	PT\_INTERP} are particularly important.
	{\ttfamily PT\_LOAD} segments are intended to be loaded into memory when 
	setting up the process. The size of the chunk at the address which it must 
	be loaded at are described in the rest of the program header.
	{\ttfamily PT\_INTERP} segments contains the {\ttfamily .interp} section 
	which provides the name of the interpreter that is to be used to load the 
	binary.
	{\ttfamily PT\_DYNAMIC} contains the {\ttfamily .dynamic} section which 
	tells the interpreter how to parse and prepare the binary for execution
	\item {\ttfamily p\_flags} contains the segment attributes. The top eight 
	bits are reserved for processor specific use and the others are reserved 
	for environment specific use.
	The possible attributes can assume the values:
	\begin{table}[!htbp]
		\begin{center}
			\begin{tabular}{|c|c|c|}
				\hline
				\textbf{Name} & \textbf{Value} & \textbf{Description}\\ 
				\hline
				{\ttfamily PF\_X} & 0x1 & Execute permission\\ \hline
				{\ttfamily PF\_W} & 0x2 & Write permission\\ \hline
				{\ttfamily PF\_R} & 0x4 & Read permission\\ \hline
				{\ttfamily PF\_MASKOS} & 3 & Environment reserved\\ \hline
				{\ttfamily PF\_MASKPROC} & 4 & Precessor reserved\\ \hline
			\end{tabular}
			\caption{Segment types}
			\label{segtypes}
		\end{center}
	\end{table}
	\item {\ttfamily p\_offset} contains the offset of the segment, in bytes, 
	from the beginning of the file;
	\item {\ttfamily p\_vaddr} contains the virtual address of the segment in 
	memory (for loadable segments this must be equal to the offset);
	\item {\ttfamily p\_paddr} is reserved for the physical address, but in 
	modern OS's is generally set to 0 (most of today's systems use virtual 
	addresses);
	\item {\ttfamily p\_filesz} and {\ttfamily p\_memsz} contain respectively 
	the size of the file image and the memory image of the segment. These are 
	different as some sections only indicate the need to allocate some bytes 
	but don't actually occupy these bytes in the bynary file (e.g. {\ttfamily 
	.bss} contains zero initialised data and since all data in this section are 
	zero there is no need to actually include them in the binary, but all of 
	its bytes are allocate when it is loaded into memory). Hence it is 
	impossible to have {\ttfamily p\_memsz} bigger than {\ttfamily p\_filesz} 
	and when this happens the loader adds extra bytes initialised to zero.
	\item {\ttfamily p\_align} specifies the alignment constraints for the 
	segment and must be a powero of two. The {\ttfamily p\_offset} and 
	{\ttfamily p\_vaddr} must be congruent modulo the alignment.
\end{itemize}
\subsection{Notes}
Sections of type {\ttfamily SHT\_NOTE} and {\ttfamily PT\_NOTE} are used by 
compilers and other tools to mark an object file with special info that has a 
special meaning for a particular toolset. These sections and segments contain 
any number of note entries, each of which is an 8-byte word. A note contains:
\begin{itemize}
	\item {\ttfamily namesz} and {\ttfamily name}. The first identifies the 
	length of a name in bytes. The latter contains a null terminated string, 
	padded to 8-byte alignment.
	\item {\ttfamily descsz} and {\ttfamily desc}. The first identifies the 
	òength of the note descripto whilst the second stores the content of the 
	note padded to 8-byte alignment as necessary.
	\item {\ttfamily type} contains a number that determines, along with the 
	oroginaor's name, the interpretation of the note contents.
\end{itemize}
\subsection{Dynamic table}
Dynamically bound object files will have a {\ttfamily PT\_DYNAMIC} program 
header entry that refers to a segment contaaining the {\ttfamily .dynamic} 
sextion. The content of the section is an array of {\ttfamily Elf64\_Dyn} 
structures as per the listing below:
\begin{lstlisting}[style=ansic, caption={Dynamic Table Structure}, 
label=elf64dyn]
	typedef struct {
        Elf64_Sxword    d_tag;
        union {
            Elf64_Xword d_val;
            Elf64_Addr  d_ptr;
        } d_un;
	} Elf64_Dyn;
\end{lstlisting}
where:
\begin{itemize}
	\item {\ttfamily d\_tag} identifies the type of dynamic table entry. The 
	type determiines the interpretation to be given to the union {\ttfamily 
	d\_un}. The processor-independent dynamic table entry types are
\begin{table}[!htbp]\begin{center}\begin{tabular}{|c|c|c|p{10cm}|}
\hline
\textbf{Name} & \textbf{Value} & \textbf{d\_un} & \textbf{Description}\\ \hline
{\ttfamily DT\_NULL} & 0 & ignored & Marks the end of the dynamic array\\
\hline
{\ttfamily DT\_NEEDED} & 1 & {\ttfamily d\_val} & The string table offset of a 
needed library\\ \hline
{\ttfamily DT\_PLTRELSZ} & 2 & {\ttfamily d\_val} & Total size, in bytes, of 
the relocation entries associated with the PLT\\ \hline
{\ttfamily DT\_PLTGOT} & 3 & {\ttfamily d\_ptr} & Address associated with the 
PLT (the specific meaning is processor dependent)\\ \hline
{\ttfamily DT\_HASH} & 4 & {\ttfamily d\_ptr} & Address of the symbol hash 
table\\ \hline
{\ttfamily DT\_STRTAB} & 5 & {\ttfamily d\_ptr} & Address of the dynamic string 
table\\ \hline
{\ttfamily DT\_SYMTAB} & 6 & {\ttfamily d\_ptr} & Address of the dynamic symbol 
table\\ \hline
{\ttfamily DT\_RELA} & 7 & {\ttfamily d\_ptr} & Address of the relocation table 
with {\ttfamily Elf64\_Rela} entries\\ \hline
{\ttfamily DT\_RELASZ} & 8 & {\ttfamily d\_val} & Total size, in bytes of the 
{\ttfamily DT\_RELA} table\\ \hline
{\ttfamily DT\_RELAENT} & 9 & {\ttfamily d\_ptr} & Size, in bytes, of the 
relocation entry\\ \hline
{\ttfamily DT\_STRSZ} & 10 & {\ttfamily d\_val} & Total size, in bytes, of the 
string table\\ \hline
{\ttfamily DT\_SYMENT} & 11 & {\ttfamily d\_val} & Size, in bytes, of each 
symbol table entry\\ \hline
{\ttfamily DT\_INIT} & 12 & {\ttfamily d\_ptr} & Address of the initialisation 
function\\ \hline
{\ttfamily DT\_FINI} & 13 & {\ttfamily d\_ptr} & Address of the termination 
function\\ \hline
{\ttfamily DT\_SONAME} & 14 & {\ttfamily d\_val} & String table offset of this 
shared object\\ \hline
{\ttfamily DT\_RPATH} & 15 & {\ttfamily d\_val} & String table offset of a 
shared library search path\\ \hline
{\ttfamily DT\_SYMBOLIC} & 16 & {\ttfamily ignored} & The presence of this 
dynamic table entry modifies the symbol resolution algorithm for references 
within the library. These are resolved befre the dynamic linker searches the 
usual search path\\ \hline
{\ttfamily DT\_REL} & 17 & {\ttfamily d\_ptr} & Address of the relocation table 
with {\ttfamily Elf64\_Rel} entries\\ \hline
{\ttfamily DT\_RELSZ} & 18 & {\ttfamily d\_val} & Size in bytes of the 
{\ttfamily DT\_REL} table\\ \hline
{\ttfamily DT\_RELENT} & 19 & {\ttfamily d\_val} & Size in bytes of the 
{\ttfamily DT\_REL} relocation entry\\ \hline
{\ttfamily DT\_PLTREL} & 20 & {\ttfamily d\_val} & Type of the relocation entry 
for the procedure linkage table. The {\ttfamily d\_val} contains either 
{\ttfamily DT\_REL} or {\ttfamily DT\_RELA}\\ \hline
{\ttfamily DT\_DEBUG} & 21 & {\ttfamily d\_ptr} & Reserved for debugger use\\ 
\hline
{\ttfamily DT\_TEXTREL} & 22 & {\ttfamily ignored} & The presence of this 
dynamic table entry signals that the relocation table contains relocation for 
non-writable segment\\ \hline
{\ttfamily DT\_JMPREL} & 23 & {\ttfamily d\_ptr} & Address of the relocations 
associated with the {\ttfamily PLT}\\ \hline
{\ttfamily DT\_BIND\_NOW} & 24 & {\ttfamily ignored} & The presence of this 
dynamic table entry signals that the dynamic loader should process all 
relocations before transferring control to th program\\ \hline
{\ttfamily DT\_INIT\_ARRAY} & 25 & {\ttfamily d\_ptr} & Pointer to array of 
pointers to the initialisation functions\\ \hline
{\ttfamily DT\_FINI\_ARRAY} & 26 & {\ttfamily d\_ptr} & Pointer to array of 
pointers to the termination functions\\ \hline
{\ttfamily DT\_INIT\_ARRAYSZ} & 27 & {\ttfamily d\_ptr} & Size in bytes of the 
array of initialisation functions\\ \hline
{\ttfamily DT\_FINI\_ARRAYSZ} & 28 & {\ttfamily d\_ptr} & Size in bytes of the 
array of termination functions\\ \hline
\end{tabular} \caption{Dynamic Table Entries} \end{center}
\end{table}
	\item {\ttfamily d\_val} union member used to represent integer values;
	\item {\ttfamily d\_ptr} union member used to represent link-time virtual 
	addresses that must be relocated to math the actual address at which the 
	object file is loaded. The relocations must be done implicitly and tthere 
	are no dynamic relocations for these items
\end{itemize} \newpage
\subsection{Hash Table}
The access to the dynamic table is efficiently guaranteed by the use of a hash 
table that is part of the loaded program segment (typically in the {\ttfamily 
.hash} section) and is pointed to by the {\ttfamily DT\_HASH} entry in the 
dynamic table. The hash table is an array of {\ttfamily Elf64\_Word} objects, 
organised as:
\begin{itemize}
	\item {\ttfamily nbucket}
	\item {\ttfamily nchain}
	\item {\ttfamily bucket[0]...bucket[nbucket - 1]}
	\item {\ttfamily chain[0]...chain[nchain - 1]}
\end{itemize}
The {\ttfamily bucket} array forms the hash table itself. The number of entries 
in the hash table is given by the first word {\ttfamily nbucket}. The entries 
in the {\ttfamily chain} reflect the symbol table. Entries of the symbol table 
are organised in hash chains, one per bucket. The hash function is listed below.
\begin{lstlisting}[style=ansic, caption={Program Header}, label=hashtab]
    unsigned long elf64_hash(const unsigned char *name) {
        unsigned long h = 0, g;
        while (*name) {
            h = (h << 4) + *name++;
            if (g = h & 0xF0000000)
                h ^= g >> 24;
            h &= 0x0FFFFFFF;
        }
        return h;
    }
\end{lstlisting}
\section{x86/x64}
The x86 is the most widely adopted processor architecture. Started in 1978 with 
the 8086 and 8088; the processor had segmentation and supported $2^{20} = 1MB$.
The 286 was issued in 1982; it had protected mode using segment registers as a 
selector with a maximum supported address space of $2^{24} = 16MB$. The game 
changer arrived in 1985 with the 386; it was the first 32-bit processor using 
virtual 8086 mode with support for an address space of $2^{32} = 4GB$ and 
paging with 4k pages. The 486 (x87), released in 1989, had an FPU. During the 
end of the 90's/beginning of the 2000 several revolutions happened:
\begin{itemize}
    \item[1993] Pentium (4k and 4M pages);
    \item[1995-99] P6 family with support for SIMD instructions and vector (MMX 
    and SSE mainly used for multimedia);
    \item[2000-2007] Pentium 4 and Xeon families, enhanced SIMD (SSE2 and 3), 
    hyperthreading and VT. This family sets a milestone as it sees the 
    introduction of the modern IA64
\end{itemize}
And now we come to 2008 with the release of the contemporary Intel Core i3, i5, 
i7 family with SSE4.2 and second generation VT.
\subsection{Modes of operation}
The operation modes, also called processor modes, CPU states or privilege level 
wok basically like gates in the execution of certain instructions. The purpose 
is to place restrictions on the type and scope of operations that can be 
performed by certain processes.
\subsubsection{32bit}
The processor boots up in real address mode (the 8086 mode) for compatibility. 
After that it reaches the protected mode where execution is carried out 
according to the privileges established for the relevant execution ring. There 
are 4 execution rings from 0 (kernel mode) to 3 (user mode). Applications run 
with a paged 32-bit flat address space. There is another mode which is the 
system management mode intended for use by firmware only.
\subsubsection{64bit}
The 64bit CPUs can run in two submodes:
\begin{enumerate}
	\item \textbf{Compatibility Mode} similar to the 32bit protected mode and 
	permits legacy 32 and 16 bit applications to run without being recompiled. 
	Allows access to $2^{36} = 64GB$ of physical memory using Physical Address 
	Extensions;
	\item \textbf{64 bit Mode} allows to run 64 bit applications, extending 
	general and SIMD registers from 8 to 16.
\end{enumerate}
\subsubsection{Memory Translation}
The CPU produces a logical address made of a 16bit Selector and the Offset; 
these are given to the GDT that returns the Base and the Limit of the LDT along 
with relevant flags. These are combined into the linear address which contains 
the Page Directory, the Page Table and the Offset. By knowing the page 
directory and the page table we can find the PPN and with the offset we 
finally find the address in memory where data and instructions stand.
MOdern OS's use paging only to provide a 32/64 bit virtual flat address space.
\subsubsection{Registers}
In 32 bit mode the segment registers CS, DS, ES and SS hold 16 bit useless 
segment selectors. In 64 bit CS, DS, ES, SS are all treated as if each segment 
base is 0 and all limit checks are disabled. FS and GS point to the TLS for 64 
and 32 bits respectively. Registers are divided in:
\begin{itemize}
	\item General Purpose
	\subitem RAX/EAX, Accumulator;
	\subitem RBX/EBX, Pointer to Data;
	\subitem RCX/ECX, Counter;
	\subitem RDX/EDX, I/O Pointer;
	\subitem RSI/ESI, Source pointer for string operations;
	\subitem RDI/EDI, Destination pointer;
	\subitem R8-R15, Only for 64 bit;
	\item Execution
	\subitem RIP/EIP, Instruction pointer;
	\subitem RSP/ESP, Stack pointer;
	\subitem RBP/EBP, Base pointer;
	\subitem RFLAGS/EFLAGS Carry, sign, zero, parity, ...
\end{itemize}
\newpage
The instructions can act on zero or more operands and data can be located:
\begin{itemize}
	\item The instruction itself in case of an immediate operand;
	\item a register
	\item a memory location
	\item an I/O port
\end{itemize}
\glsaddall
\printglossaries
\end{document}